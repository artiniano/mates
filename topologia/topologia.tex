\documentclass[12pt]{article}
\usepackage[utf8]{inputenc}
\usepackage[spanish]{babel}
\usepackage{amsmath}
\usepackage{amsfonts}
\usepackage{amssymb}
\usepackage{amsthm}
\usepackage{blindtext}
\usepackage{mathtools}
\usepackage{graphicx}
\usepackage{latexsym}
\usepackage{cancel}
\usepackage[left=2cm,top=2cm,right=2cm,bottom=2cm]{geometry}
\usepackage[all]{xy}
\usepackage{cancel}
\usepackage{pictexwd}
\usepackage{parskip}
\usepackage{pgfplots}
\pgfplotsset{compat=1.15}
\usepackage{mathrsfs}
\usepackage{vmargin}


\DeclarePairedDelimiter\Floor\lfloor\rfloor
\DeclarePairedDelimiter\Ceil\lceil\rceil


\newtheorem{theorem}{Teorema}[section]
\newtheorem{definicion}[theorem]{Definición}
\newtheorem{proposition}[theorem]{Proposición}
\newtheorem{lemma}{Lema}[theorem]
\newtheorem{definition}[theorem]{Definición}
\newtheorem{example}{Ejemplo}[theorem]
\newtheorem{corolario}{Corolario}[theorem]
\newtheorem{observation}{Observación}[theorem]
\newtheorem{properties}{Propiedades}[theorem]
\providecommand{\abs}[1]{\lvert#1\rvert}
\providecommand{\norm}[1]{\lVert#1\rVert}


\author{Pablo Pallàs}
\title{Topología general}
\setlength{\parindent}{10pt}


\begin{document}
\rmfamily
\maketitle
\tableofcontents
\parindent= 0cm

\section{Espacios topológicos}
\subsection{Espacios topológicos}
Sea $X$ un conjunto y $\mathcal{P}(X) = \lbrace A : A \subset X \rbrace$ el conjunto de sus partes, entonces:

\begin{definition}Una \textbf{topología} sobre un conjunto $X$ es un subconjunto $\tau \subseteq \mathcal{P}(X)$ que satisface: 
 \renewcommand{\theenumi}{\roman{enumi}} %Números arábigos
\begin{enumerate}
\item El conjunto vacío $\emptyset$ y el conjunto total $X$ pertenecen a $\tau$.
\item La unión arbitraria de elementos de $\tau$ también pertenece a $\tau$.
\item La intersección finita de elementos de $\tau$ también pertenece a $\tau$.
\end{enumerate}
El par $(X,\tau)$ lo denominaremos \textbf{espacio topológico} y a los elementos de $\tau$ los llamaremos \textbf{abiertos}.
\end{definition}

Es decir, podríamos decir que una topología es una colección de subconjuntos que contiene al vacío y al total, y que es cerrada para las uniones arbitrarias y las intersecciones finitas.

\begin{example}Sea $X$ un conjunto arbitrario y $\tau_D = \mathcal{P}(X)$. Entonces, $\tau_D$ es una topología en $X$ ya que contiene a todos los subconjuntos de $X$, en particular al vacío y al total, es cerrada para las uniones arbitrarias y para las intersecciones finitas. A esta topología la denominaremos \textbf{topología discreta}, y al conjunto $X$ dotada de esta topología \textbf{espacio discreto}.
\end{example}
\begin{example}Sea $X$ un conjunto arbitrario y $\tau_I = \lbrace \emptyset, X \rbrace$. Entonces la colección $\tau_I$ es una topología sobre $X$: contiene al vacío y al total, la unión de ambos es $X \in \tau_I$ y la intersección es $\emptyset \in \tau_I$. Esta topología la denominaremos \textbf{topología indiscreta}, y es la topología más simple que puede tener un conjunto. A un conjunto $X$ dotado con esta topología lo denominaremos \textbf{espacio indiscreto}.
\end{example}

\begin{definition}Dos topologías $\tau_1, \tau_2$ sobre un conjunto $X$ se dicen \textbf{comparables} si $\tau_1 \subset \tau_2$ ó $\tau_2 \subset \tau_1$. Si $\tau_1 \subset \tau_2$ diremos que $\tau_2$ es más \textbf{fina} (tiene más abiertos) que $\tau_1$.
\end{definition}

Intuitivamente podríamos decir que una topología $\tau'$ es más fina que otra $\tau$ si tiene todos los abiertos de $\tau$ y añgunos más. Una topología más fina distingue de forma ''más fina'' los puntos y sus alrededores. Evidentemente, sobre un conjunto $X$ cualquiera la topología más fina que podemos encontrar es la topología discreta $\tau_D$. Por otra parte, la topología indiscreta $\tau_I$ es la menos fina que podemos encontrar. Luego cualquier otra topología $\tau$ se encontrará entre estas dos: $\tau_I \subseteq \tau \subseteq \tau_D$.

Notar que si $\tau_1$ y $\tau_2$ son dos topologías sobre $X$ es fácil ver que $\tau_1 \cap \tau_2$ es una topología sobre $X$. En general, la unión $\tau_1 \cup \tau_2$ no es necesariamente una topología.

\begin{example}\label{eq:topUsual} Consideremos el siguiente conjunto: 

$$\tau_u = \lbrace U \subset \mathbb{R}: \forall x \in U\hspace{0.1cm} \exists \epsilon > 0\hspace{0.1cm} \text{t.q}\hspace{0.1cm} (x-\epsilon, x+ \epsilon) \subset U \rbrace.$$

Entonces: 
\begin{enumerate}
\item $\emptyset \in \tau_u$ trivialmente y $\mathbb{R} \in \tau_u$ ya que si tenemos un $x \in \mathbb{R}$ y $\epsilon = 1$, entonces $(x-1, x+1) \in \mathbb{R}.$
\item Dada $\lbrace U_i \rbrace_{i\in J}$ una colección arbitraria de elementos de $\tau_u$ entonces,  si consideramos un $x \in \cup_i U_i$ existirá un $i_0 \in J$ tal que $x \in U_{i_0}$. Como $U_{i_0} \in \tau_u$ existirá un $\epsilon >0$ tal que $(x-\epsilon, x + \epsilon) \subset U_{i_0} \subset \cup_i U_i$ y así $\cup_i U_i \in \tau_u$.
\item Sean $U, V \in \tau_u$, $x \in U \cap V$. Como $U \in \tau_u$ existirá un $\epsilon_1>0$ tal que $(x-\epsilon_1, x+\epsilon_1) \subset U$. Como $V \in \tau_u$ existirá un $\epsilon_2 >0$ tal que $(x-\epsilon_2, x+\epsilon_2) \subset V$. Ahora, si escogemos $\epsilon = \min \lbrace \epsilon_1, \epsilon_2 \rbrace$ entonces tendremos: $$(x-\epsilon, x+\epsilon) \subset U \subset U \cap V$$ $$ (x-\epsilon, x+\epsilon) \subset V \subset U \cap V$$ y así $ U \cap V \in \tau_u$.
\end{enumerate}

Luego $\tau_u$ es una topología, que denominaremos \textbf{topología usual}. Al espacio topológico $(\mathbb{R}, \tau_u)$ lo denominaremos \textbf{recta real}.
\end{example}

$\hfill \blacksquare$

\begin{definition}Sea $(X, \tau)$ un espacio topológico, un conjunto $\mathfrak{B} \subset \tau$ de abiertos se dice \textbf{base} de $\tau$ si todo elemento de $\tau$ es unión de elementos de $\mathfrak{B}$. A estos elementos de $\mathfrak{B}$ los denominaremos \textbf{abiertos básicos}.
\end{definition}

\begin{example}Veamos algunos ejemplos:
\begin{enumerate}
\item La propia topología $\tau$ es base de sí misma.
\item Es claro que $\mathfrak{B} = \lbrace \lbrace x \rbrace : x \in X \rbrace$ es base de la topología discreta $\tau_D$ sobre $	X$.
\item El conjunto de intervalos $\mathfrak{B}_U = \lbrace (a,b):a,b \in \mathbb{R} \rbrace$ es una base para la topología usual sobre $\mathbb{R}$, $\tau_U.$
\end{enumerate}
\end{example}

\begin{proposition}Sea $(X, \tau)$ un espacio topológico, entonces $\mathfrak{B} \subset \tau$ es una base si y sólo si para todo $U \in \tau$ y todo $x \in U$ existe $B \in \mathfrak{B}$ tal que $x \in B \subset U$.
\end{proposition}
\emph{Demostración: }Sea $x \in U$, si $\mathfrak{B} = \lbrace B_i : B_i \tau \rbrace $ es una base entonces $U \cup B_i$, por lo que existirá $B_k \in \mathfrak{B}$ tal que $x \in B_k \subset U$. Recíprocamente, dado $U \in \tau$, si para todo $x_i \in U$ existe $B_i \in \mathfrak{B}$ tal que $x_i \in B_i \subset U$ entonces es claro que $U = \cup B_i$.

$\hfill \square$

\begin{proposition}Sea $(X,\tau)$ un espacio topológico y $\mathfrak{B} \subset \tau$ una base, entonces $U \subset X$ es un abierto si y sólo si para todo $x \in U$ existe un $B \in \mathfrak{B}$ tal que $x \in B \subset U$.
\end{proposition}

Luego:

\begin{corolario}Sea $(X, \tau)$ un espacio topológico y $A \subset X$, entonces $A$ es abierto si y sólo si para todo $x \in A$ existe un abierto $U \in \tau$ tal que $x \in U \subset A$.
\end{corolario}

Sin embargo, no toda familia de partes de un conjunto es una base para una topología. Para identificar a estos conjuntos especiales tenemos el siguiente resultado:

\begin{proposition}\label{eq:condBase} Sea $\mathfrak{B} \subset \mathcal{P}(X)$ satisfaciendo: 
\begin{enumerate}
\item Para todo $x \in X$ existe un $B \in \mathfrak{B}$ tal que $x \in B$.
\item Dados $B_1, B_2 \in \mathfrak{B}$ y $x \in B_1 \cap B_2$, entonces existe un $B \in \mathfrak{B}$ tal que $x \in B \subset B_1 \cap B_2$
\end{enumerate}
Entonces, el conjunto $\tau_{\mathfrak{B}} \subset \mathcal{P}(X)$ definido por $U \in \tau_{\mathfrak{B}}$ si y sólo si para todo $x \in U$ existe $B \in \mathfrak{B}$ tal que $x \in B \subset U$ es una topología sobre $X$ que tiene a $\mathfrak{B}$ como base. Llamaremos a $\tau_{\mathfrak{B}}$ topología generada por $\mathfrak{B}$.
\end{proposition}
\begin{proof}Trivialmente se tiene que $\emptyset \in \tau_{\mathfrak{B}}$ y también está claro que $X \in \tau_{\mathfrak{B}}$. 

Sea ahora $\lbrace U_i\rbrace_{i\in J} \subset \tau_{\mathfrak{B}}$ y $U = \cup_i U_i$, dado un $x \in U$ entonces $x \in U_k$ para algún $k \in J$ y existirá $B \in \mathfrak{B}$ tal que $x \in B \subset U_k \subset U$, por lo que $U \in \tau_{\mathfrak{B}}$. 

Sean ahora $U_1$ y $U_2 \in \tau_{\mathfrak{B}}$ y veamos que $U_1 \cap U_2 \in \tau_{\mathfrak{B}}$, en efecto, dado $x \in U_1 \cap U_2$ existirán $B_1, B_2 \in \mathfrak{B}$ tales que $x \in B_1 \subset U_1$ y $x \in B_2 \subset U_2$, entonces $x \in B_1 \cap B_2$ y así existirá un $B \in \mathfrak{B}$ tal que $x \in B \subset B_1 \cap B_2 \subset U_1 \cap U_2$, por lo que $U_1 \cap U_2 \in \tau_{\mathfrak{B}}$. 

Por inducción finita se sigue para cualquier subfamilia finita $\lbrace U_1, \ldots U_n \rbrace \subset \tau_{\mathfrak{B}}$. Se concluye así que $\tau_{\mathfrak{B}}$ es una topología sobre $X$ con base $\mathfrak{B}$.

\end{proof}

\begin{example}Consideremos el producto cartesiano $\mathbb{R}^n$ de $n$ copias del conjunto de los números reales. En $\mathbb{R}^n$ consideramos la \textbf{distancia euclídea} $$
\begin{array}{rccl}
d \colon &\mathbb{R}^n \times \mathbb{R}^n & \longrightarrow & \mathbb{R}^n\\
&((x_1, \ldots, x_n),(y_1, \ldots, y_n))& \longmapsto &\sqrt{(x_1-y_1)^2+\ldots+(x_n-y_n)^2}
\end{array}
$$

Entonces, dado $x \in \mathbb{R}^n$ y $r \in \mathbb{R}$ con $r>0$, podemos definir la \textbf{bola abierta} de centro $x$ y radio $r$ como el subconjunto $$B(x,r) = \lbrace y \in \mathbb{R}^n: d(x,y) <r \rbrace.$$ La colección de bolas en $\mathbb{R}^n$ $$\mathfrak{B} = \lbrace B(x,r) :x \in \mathbb{R}^n, r>0 \rbrace$$ forman una base para la topología usual en $\mathbb{R}^n$. A $\mathbb{R}^n$ dotado de esta topología se denonima \textbf{espacio euclídeo $n$-dimensional}. Concretamente, la recta real $(\mathbb{R}, \tau_u)$ es el espacio euclídeo $1$-dimensional.

Notar: 
\begin{enumerate}
\item $\mathbb{R}^n = \cup_x B(x,1)$.
\item Sean $B_1, B_2 \in \mathfrak{B}$ tales que $B_1 \cap B_2 \neq \emptyset$. Supongamos que $B_1 = B(x_1,r_1)$ y $B_2 = B(x_2,r_2)$ y sea $x \in B_1 \cap B_2$. Tenemos que ver que existe un $r>0$ tal que $B(x,r) \subset B_1 \cap B_2$. Para ello basta escoger $0 < r< min\lbrace	r_1-d(x,x_1), r_2-d(x,x_2)\rbrace$. Así, si $y \in B(x,r)$ entonces $d(y,x_i) \leq d(y,x) + d(x,x_i) <r + d(x,x_i)<r_i$, con $i=1,2$.
\end{enumerate}

Realmente, como ya sabemos, no necesitamos definir una distancia para encontrar una base para un espacio topológico ni tampoco para definirlo. Esto se utilizará más adelante para hablar de otro tipo de espacios.
\end{example}

$\hfill \blacksquare$

\begin{proposition}Sean $\mathfrak{B}_1$ y $\mathfrak{B}_2$ bases de sendas topologías $\tau_1 $y $\tau_2$ sobre un conjunto $X$, entonces $\tau_2$ es más fina que $\tau_1$, es decir, $\tau_1 \subset \tau_2$ si y sólo si para todo $x\in X$ y todo $B_1 \in \mathfrak{B}_1$ tal que $x\in B_1$ existe $B_2 \in \mathfrak{B}_2$ tal que $x \in B_2 \subset B_1.$
\end{proposition}
\begin{proof}Supongamos que $\tau_1 \subset \tau_2$, dado $x \in X$ y $B_1\in \mathfrak{B}_1$ tal que $x \in B_1$, en particular $B_1 \in \tau_1$ y por tanto $B_1 \in \tau_2$, entonces existe $B_2 \in \mathfrak{B}_2$ tal que $x \in B_2 \subset B_1$.

Recíprocamente, si $U \in \tau_1$ existirá $B_1 \in \mathfrak{B}_1$ tal que $x \in B_1 \subset U$ y como existe $B_2 \in \mathfrak{B}_2$ tal que $x \in B_2 \subset B_1$ se tiene entonces que $x \in B_2 \subset U$ y por tanto se sigue que $U \in \tau_2$.

\end{proof}

\begin{example}
La familia $\mathfrak{B}_S$ de los intervalos semiabiertos de la forma $[a,b)$ satisfacen las condiciones de~\ref{eq:condBase}, luego forman base de una topología $\tau_S$ sobre $\mathbb{R}$. Al espacio topológico $(\mathbb{R},\tau_S)$ ó simplemente $\mathbb{R}_S$ se le conoce como \textbf{recta de Sorgenfrey}.

Notar que para todo $x \in \mathbb{R}$ y todo intervalo $(a,b)$ tal que $x \in (a,b)$ está claro que $x \in [x,b) \subset (a,b)$, por lo que $\tau_U \subset \tau_S$, y como $[a,b) \notin \tau_U$ se sigue que la topología de Sorgenfrey es estrictamente más fina que la usual.
\end{example}

$\hfill \blacksquare$

\begin{definition}Dado un espacio topológico $(X,\tau)$ y un punto $x \in X$, llamaremos \textbf{entorno} de $x$ a todo subconjunto $N \subset X$ que contiene a $x$ como punto interior, es decir, a todo $N$ para el cual existe $U \in \tau$ tal que $x \in U \subset N$.
\end{definition}

Aunque, en general, los entornos no son necesariamente abiertos, nosotros solamente consideraremos entornos abiertos. Así, por entorno de un punto entenderemos todo abierto que lo contiene.

\begin{definition}Dado un espacio topológico $(X, \tau)$, denotaremos por $\varepsilon(x) = \lbrace U \in \tau: x \in U \rbrace$ a la familia de todos los entornos abiertos de $x$ y diremos que $\mathfrak{B}_x \subset \varepsilon (x)$ es una \textbf{base de entornos de $x$} ó \textbf{base local respecto de $x$} si para todo $U \in \tau$ tal que $x \in U$ existe un $B \in \mathfrak{B}_x$ tal que $x \in B \subset U$.
\end{definition}

Además, está claro que $\mathfrak{B} = \cup_x \mathfrak{B}_x$ es una base para $\tau$. Recíprocamente, si $\mathfrak{B}$ es una base para $\tau$, entonces $\mathfrak{B}_x = \mathfrak{B} \cap \varepsilon(x)$ es una base local respecto de $x$.

No toda familia de partes de un conjunto $X$ es una topología, ni forman base para una topología. Sin embargo, todo subconjunto $\mathcal{S} = \lbrace S_{ij} \rbrace \subset \mathcal{P}(X)$ tal que $X = \cup S_{ij}$ determina una topología sobre $X$: si denotamos por $\mathfrak{B}_{\mathcal{S}} = \lbrace B_i \rbrace$ el conjunto de las intersecciones finitas de elementos de $\mathcal{S}$, es decir, $B_i = S_{1i} \cap \ldots \cap S_{ni}$ para algún $n$, entonces el conjunto $\tau_{\mathcal{S}}$ de las uniones arbitrarias de elementos de $\mathfrak{B}_{\mathcal{S}}$ es una topología sobre $X$ que tiene a $\mathfrak{B}_{\mathcal{S}}$ como base. Notaremos que $\tau_{\mathcal{S}}$ es la menor topología sobre $X$ tal que $\mathcal{S} \subset \tau_{\mathcal{S}}$. El conjunto $\mathcal{S}$ se dirá \textbf{\textit{subbase}} de $\tau_{\mathcal{S}}$.

\begin{definition}Sea $(X, \tau)$ un espacio topológico y $A \subset X$ un subconjunto de $X$. Un punto $x \in A$ se dice \textbf{punto interior} de $A$ si existe un abierto $U \in \tau$ tal que $x \in U \subset A$. Denotaremos por $Int A$ el conjunto de todos los puntos interiores de $A$. Claramente $Int A \subset A$ y $A$ es abierto si y sólo si $A = Int A$.
\end{definition}

\begin{proposition}Sea $(X,\tau)$ un espacio topológico y $A \subset X$. El conjunto $Int A$ de todos los puntos interiores de $A$ es la unión de todos los abiertos contenidos en $A$. En particular, $Int A$ es el mayor abierto contenido en $A$.
\end{proposition}
\begin{proof}
Sea $\mathcal{U} = \lbrace U\in \tau: U \subset A \rbrace$ la familia de todos los abiertos contenidos en $A$, todo punto de $U$ es interior y por tanto $U \subset Int A$ para todo $U \in \mathcal{U}$, por lo que se sigue que $\cup_{U \in \mathcal{U}} U \subset Int A$. Por otro lado, para todo $x \in Int A$ existe un abierto $U_x \in \tau$ tal que $x \in U_x \subset A$, por lo que $Int A = \cup_{x\in Int A} \lbrace x \rbrace \subset \cup_{x\in Int A}U_x \subset \cup_{U \in \mathcal{U}} U$, ya que $U_x \in \mathcal{U}$ y así concluimos que $Int A = \cup_{U \in \mathcal{U}} U$.

\end{proof}

\begin{proposition}Sea $(X, \tau)$ un espacio topológico y $A \subset X$, entonces se satisface:
\begin{enumerate}
\item $Int(Int A) = Int A$.
\item Si $B \subset A$ entonces $Int B \subset Int A$.
\item $Int (A \cap B) = Int A \cap Int B$ $\forall A,B \in X$.
\item $Int A \cup Int B \subset Int (A \cup B)$ $\forall A,B \in X$.
\end{enumerate}
\end{proposition}
\begin{proof}
Son todas inmediatas. Observar que en el último caso el contenido es estricto, por ejemplo: en $\mathbb{R}$ con la topología usual tomamos $A=[0,1]$ y $B = [1,2]$, entonces $A \cup B = [0,2]$ y tenemos $Int(A \cup B) = (0,2)$, sin embargo $Int A \cup Int B = (0,1) \cup (0,2) = (0,2) \setminus \lbrace 1 \rbrace.$

\end{proof}

\begin{definition}Sea $(X, \tau)$ un espacio topológico, un subconjunto $F \subset X$ se dirá \textbf{cerrado} si su complementario, $X \setminus F$, es abierto.
\end{definition}

Aplicando las leyes de De Morgan obtenemos el siguiente resultado: 

\begin{proposition}Sea $(X, \tau)$ un espacio topológico, la familia de los cerrados satisface:\begin{enumerate}
\item $X$ y $\emptyset$ son cerrados.
\item La intersección arbitraria de cerrados es un cerrado.
\item La unión finita de cerrados es un cerrado.
\end{enumerate}
\end{proposition}

Y es que en un espacio topológico, abiertos y cerrados se determinan mutuamente. Dada una familia $\mathcal{F} \subset \mathcal{P}(X)$ de subconjuntos de $X$ satisfaciendo las propiedades anteriores, entonces es claro que $\tau = \lbrace X \setminus F: F \in \mathcal{F} \rbrace$ es una topología sobre $X$ tal que $\mathcal{F}$ es la familia de sus cerrados.

\begin{definition}Sea $(X, \tau)$ un espacio topológico y $A \subset X$, un punto $x \in X$ se dice \textbf{punto de acumulación de $A$} si para todo abierto $U$ tal que $x \in U$ se tiene que $(U \setminus \lbrace x \rbrace ) \cap A \neq 0$. Llamaremos \textbf{derivado de $A$} y lo denotaremos por $A'$ al conjunto de todos los puntos de acumulación de $A$.
\end{definition}

Además, dados $A,B \subset X$, entonces $(A \cup B)' = A' \cup B'$. En particular, si $A \subset B$ entonces $A' \subset B'$.

Los cerrados son aquellos que contienen a su derivado:

\begin{proposition}Sea $(X, \tau)$ un espacio topológico y $F \subset X$, entonces $F$ es cerrado si y sólo si $F' \subset F$.
\end{proposition}
\begin{proof}
Sea $F$ cerrado y $x \in F'$, si $x\notin F$ entonces $x \in X \setminus F$ abierto y así existirá un abierto $U \in \tau$ tal que $x \in U \subset X \setminus F$, en particular $(U \setminus \lbrace x \rbrace) \cap F = \emptyset$, lo cual contradice que $x \in F'$.

Recíprocamente, si $F' \subset F$ y $x \in X \setminus F$ entonces $x \notin F'$ y existirá un $U$ abierto tal que $(U \setminus \lbrace x \rbrace) \cap F = \emptyset$ ó bien $U \cap F = \emptyset$, ya que $x \notin F$. Así, $x \in U \subset X \setminus F$, es decir, $X\setminus F$ es abierto y $F$ es cerrado.

\end{proof}
\begin{definition}Sea $(X, \tau)$ un espacio topológico y $A \subset X$, definimos la \textbf{clausura} o cierre de $A$ en $X$, y lo denotamos por $Cl_X A$ o $\overline{A}$ a la unión de $A$ con su derivado, es decir, $\overline{A} = A \cup A'$.
\end{definition}

Además, está claro que $A \subset \overline{A}$ para todo $A \subset X$ y es claro que $A$ es cerrado si y sólo si $A = \overline{A}$.

Los puntos de la clausura, denominados \textbf{\textit{puntos adherentes}} o de adherencia, se pueden caracterizar así:

\begin{proposition}Sea $(X, \tau)$ un espacio topológico y $A \subset X$, entonces $x \in \overline{A}$ si y sólo si $U \cap A \neq \emptyset$ para todo $U \in \tau$ tal que $x \in U$.
\end{proposition}

\begin{proposition}Sea $(X, \tau)$ un espacio topológico y $A,B \subset X$, entonces se satisface: \begin{enumerate}
\item $X \setminus \overline{A} = Int(X\setminus A)$.
\item $\overline{X\setminus A} = X \setminus Int A$.
\item $\overline{A \cup B} = \overline{A} \cup \overline{B},$ en particular $A \subset B \Rightarrow \overline{A} \subset \overline{B}.$
\item $\overline{A \cap B} \subset \overline{A} \cap \overline{B}.$
\end{enumerate}
\end{proposition}
\begin{proof}
Notar que $x \notin \overline{A}$ si y sólo si existe $U \in \tau$ tal que $x \in U \subset X \setminus A$, por lo que $X \setminus \overline{A} = Int (X \setminus A)$. Claramente $2$ se sigue de $1$ cambiando $A$ por $X\setminus A$. Por $1$, $\overline{A \cup B} = X \setminus Int[X \setminus (A \cup B)] = X \setminus Int[(X\setminus A)\cap (X\setminus B)] = X \setminus [Int(X \setminus A) \cap Int(X \setminus B)] = [X \setminus Int(X\setminus A) ]\cup [X \setminus Int(X\setminus B)] = \overline{A} \cup \overline{B}$. Si $A \subset B$ entonces $B = A \cup B$ y así $\overline{B} = \overline{A \cup B} = \overline{A} \cup \overline{B}$, y en consecuencia $\overline{A} \subset \overline{B}$. De esto último deducimos fácilmente que $\overline{A \cap B} \subset \overline{A} \cap \overline{B}$.7

\end{proof}

A lo largo de la demostración anterior se utiliza una expresión que se obtiene de reformular el primer punto. Y es que podemos deducir claramente que $\overline{A} = X \setminus Int(X\setminus A).$

Además, el contenido del último punto es, en general, estricto. En efecto, sea $\mathbb{R}$ con la topología usual, $A = \mathbb{Q}$ y $B = \mathbb{R} \setminus \mathbb{Q}$, entonces para todo $x \in \mathbb{R}$ es claro que $(x-r, x+r) \cap \mathbb{Q} \neq \emptyset$ para todo $r >0$, por lo que $x \in \overline{\mathbb{Q}}$, es decir, $\overline{\mathbb{Q}} = \mathbb{R}.$ Análogamente, $\overline{\mathbb{R} \setminus \mathbb{Q}} = \mathbb{R}$, por lo que $\overline{A} \cap \overline{B} = \mathbb{R}$ mientras que $\overline{A \cap B} = \overline{\emptyset} = \emptyset.$

\begin{proposition}La clausura de $A$ es la intersección de todos los cerrados que contienen a $A$. En particular, $\overline{A}$ es el menor cerrado que contiene a $A$.
\end{proposition}
\begin{proof}
Sea $\mathcal{F} = \lbrace F_i : F_i = \overline{F_i}, A \subset F_i \rbrace$ la familia de los cerrados que contienen a $A$, notar que $\overline{A} \in \mathcal{F}$, luego es claro que $\cap F_i  \subset \overline{A}$. Por otra parte, $A \subset F_i$ implica $\overline{A} \subset \overline{F_i} = F_i$ para todo $F_i \in \mathcal{F}$ y así $\overline{A} \subset \cap F_i$.

\end{proof}

\begin{definition}Sea $(X, \tau)$ un espacio topológico. Definimos la \textbf{frontera} de $A \subset X$, como $Fr A = \overline{A} \cap \overline{X\setminus A}$. En particular, $FrA$ es un cerrado para todo $A \subset X$ y es claro que $FrA = Fr(X\setminus A)$.
\end{definition}

Notar que si $A$ es abierto y cerrado entonces $Fr A = \emptyset$.

Los puntos de la frontera los podemos caracterizar como sigue:

\begin{proposition}Sea $(X, \tau)$ un espacio topológico. Sea $A \subset X$, entonces $x \in Fr A$ si y sólo si $U \cap A \neq \emptyset$ y $U \cap (X \setminus A) \neq \emptyset$ para todo $U\in \tau$ tal que $x \in U$, es decir, todo entorno de $x$ corta a $A$ y a su complementario.
\end{proposition}

\begin{proposition}Sea $(X, \tau)$ un espacio topológico y $A \subset X$. Entonces: \begin{enumerate}
\item $Fr A = \overline{A} \setminus Int A$.
\item $\overline{A} = Int A \sqcup Fr A$.
\item Si $FrA \cap Fr B = \emptyset$ entonces $Int A \cup Int B = Int(A \cup B)$, $\overline{A \cap B} = \overline{A} \cap \overline{B}$ y $Fr(A \cap B) = (Fr A\cap \overline{B})\cup (\overline{A} \cap Fr B)$.
\end{enumerate}
\end{proposition}
\begin{proof}
Sólo el último punto lleva cierta dificultad. Sea $Fr A \cap Fr B = \emptyset$ y supongamos que existe un $x \in Int (A \cup B)$ tal que $x \notin Int A \cup Int B$, entonces $x \in X \setminus (Int A \cup Int B) = (X \setminus Int A) \cap (X \setminus Int B) = \overline{X\setminus A} \cap \overline{X\setminus B}$. Si $x \notin \overline{A}$ existirá $U \in \tau$ tal que $x \in U$ y $U \cap A = \emptyset$. Por otro lado, $x \in Int (A \cup B)$ implica que existe $V \in \tau$ tal que $x \in V \subset A \cup B$. Entonces $U \cap V \in \tau$ satisface $x \in U \cap V \subset V \subset A \cup B$ y $(U \cap V) \cap A \subset U \cap A = \emptyset$, lo cual implica que $x \in U \cap V \subset B$, es decir, $x \in Int B$ y llegamos a una contradicción ya que $x \notin Int A \cup Int B$. Necesariamente tenemos que $x \in \overline{A}$ y análogamente podríamos ver que $x \in \overline{B}$. Luego $x \in \overline{A} \cap \overline{X \setminus A} \cap \overline{B} \cap \overline{X \setminus B} = Fr A \cap Fr B$, lo que contradice la hipótesis de que las fronteras de $A$ y $B$ son disjuntas. Necesariamente tenemos que $x \in Int A \cup Int B$ y por tanto $Int (A \cup B) \subset Int A \cup Int B$. El otro contenido se satisface sin restricciones y tenemos la igualdad. Por otra parte, notar que $X \setminus \overline{A \cap B} = Int (X \setminus A \cap B) = Int[(X\setminus A) \cup (X \setminus B)] = Int (X\setminus A ) \cup Int (X \setminus B)$, ya que $Fr (X \setminus A) \cap Fr(X\setminus B) = Fr A \cap Fr B = \emptyset$, pero $Int (X\setminus A) \cup Int (X\setminus B) =(X\setminus \overline{A})\cup (X  \setminus \overline{B}) = X \setminus (\overline{A} \cap \overline{B})$ y así $\overline{A \cap B} = \overline{A} \cap \overline{B}$. Finalmente, tenemos que $Fr (A \cap B) = \overline{A \cap B} \cap \overline{X \setminus (A \cap B)} = \overline{A} \cap \overline{B} \cap (\overline{X\setminus A} \cup \overline{X \setminus B}) = (\overline{A}\cap \overline{B} \cap \overline{X\setminus A})\cup (\overline{A} \cap \overline{B} \cap \overline{X \setminus B}) = (Fra A \cap \overline{B})  \cup (\overline{A} \cap FrB).$

\end{proof}

\begin{definition}Sea $(X, \tau)$ un espacio topológico y $A \subset X$. Definimos el \textbf{exterior} de $A$ como el interior de su complementario. Así, $Ext A = Int (X \setminus A)$.  Notar que $X = Int A \sqcup Fr A \sqcup Ext A$ para todo $A \subset X$. Un punto $x \in A$ se dirá aislado si existe un abierto $U$ en $X$ tal que $U \cap A = \lbrace x \rbrace$. Denotaremos por $Ais(A)$ el conjunto de los puntos aislados de $A$. Notar que $\overline{A} = Ais(A) \sqcup A'$, en particular si $A' = \emptyset$ entonces todo punto de $A$ es aislado.
\end{definition}

\begin{definition}Sea $(X, \tau)$ un espacio topológico. Un subconjunto $D \subset X$ se dice \textbf{denso} en $X$ si $\overline{D} = X$, o equivalentemente si para todo abierto $U$ en $X$ se tiene $U \cap D \neq \emptyset$.
\end{definition}

Tanto los racionales como los irracionales son densos en $\mathbb{R}$.

Veamos ahora un ejemplo en el que repasemos todos los distintos tipos de puntos en un espacio topológico:

\begin{example}Sea $(\mathbb{R}, \tau_u)$ la recta real y $A = (0,1] \subset (\mathbb{R}, \tau_u)$.
\begin{enumerate}
\item El conjunto de los puntos interiores, $Int A = (0,1)$. En efecto, para cualquier $x \in (0,1)$ existe un $\varepsilon >0$ tal que $x \in (x-\varepsilon, x+ \varepsilon) \subset (0,1)$. Y cualquier otro punto de la recta, incluido el $1$, no cumple esta condición.
\item La clausura de $A$, $\overline{A} = [0,1]$. Dado $x \in [0,1]$ y un abierto $(a,b)$ tal que $x \in (a,b)$, siempre se va a tener que $(a,b) \cap (0,1] \neq \emptyset$. Sin embargo, si $x \notin [0,1]$ siempre podemos encontrar un abierto $(a,b)$ tal que $x \in (a,b)$ y $(a,b) \cap [0,1] = \emptyset$.
\item El derivado de $A$, $A' = [0,1]$. Dado cualquier $x \in [0,1]$ y un abierto $(a,b)$ que lo contenga, siempre se tiene que $((a,b) \setminus \lbrace x \rbrace ) \cap (0,1] \neq \emptyset$. Por el contrario, si $x \notin [0,1]$ siempre podemos encontrar un abierto $(a,b)$ que lo contenga y $((a,b) \setminus \lbrace x \rbrace ) \cap [0,1] \neq \emptyset$.
\item El conjunto de los puntos aislados de $A$, $Ais(A) = \emptyset$. Y es que no existe ningún punto $x \in \mathbb{R}$ tal que un abierto $(a,b)$ que lo contenga interseque a $(0, 1]$ únicamente en el punto $x$.
\item La frontera de $A$, $Fr A = \lbrace 0, 1\rbrace$. Los únicos puntos $x$ tales que cualquier abierto $(a,b)$ con $x \in (a,b)$ que verifican que intersecan tanto a $(0,1]$ como a sus complementarios $\mathbb{R}\setminus (0,1]$ son el $0$ y el $1$.
\end{enumerate}
\end{example}

$\hfill \blacksquare$

\begin{example}Si consideramos ahora como espacio topológico $\mathbb{Z}$ dotado de la topología usual entonces, dado un entero $z \in \mathbb{Z}$, existe un intervalo, por ejemplo $z \in (z-1, z+1)$, que corta a $\mathbb{Z}$ en únicamente el punto $z$. Así, todos los puntos de $\mathbb{Z}$ son aislados, frontera y los únicos que conforman la clausura, ya que para todo $x \in \mathbb{R}\setminus \mathbb{Z}$ se tiene que $x \in (\lfloor x \rfloor, \lfloor x \rfloor + 1) \subset \mathbb{R} \setminus \mathbb{Z}$ (recordemos que la función suelo $\lfloor x \rfloor = max\lbrace k \in \mathbb{Z}: k\leq x \rbrace$), luego tampoco hay puntos interiores ni de acumulación. Así, tenemos que $\overline{\mathbb{Z}} = Ais (\mathbb{Z}) = Fr\mathbb{Z} = \mathbb{Z}$ y $Int\mathbb{Z} = \mathbb{Z}' = \emptyset$.

Notar que el conjunto de los puntos interiores es vacío porque todo abierto que contenga a un punto $z \in \mathbb{Z}$ será de la forma $(z-\varepsilon, z+\varepsilon)$ y este abierto siempre pertenece a $\mathbb{R}\setminus \mathbb{Z}$.
\end{example}

$\hfill \blacksquare$

\section{Aplicaciones continuas y homeomorfismos}

\subsection{Introducción y homeomorfismos}

Una vez hemos estudiado lo que es una topología y las principales propiedades de los espacios topológicos veremos las aplicaciones que preservan su estructura: las \textit{aplicaciones continuas}. Cuando hablamos de preservar la estructura en topología nos referiremos a conservar la idea de \textit{proximidad} entre puntos. Es decir, que estas aplicaciones llevarán puntos cercanos en puntos cercanos. 

Antes de pasar a hablar de continuidad en espacios topológicos recordemos brevemente cómo se definía la continuidad de una función real de variable real. Dada una función $f \colon \mathbb{R} \longrightarrow \mathbb{R}$ y un punto $x \in \mathbb{R}$, se dice que $f$ es \textit{continua} en $x$ si para cada $\varepsilon >0$ existe $\delta >0$ tal que si $y \in (x- \delta, x + \delta)$ entonces $f(y) \in (f(x)-\varepsilon, f(x) + \varepsilon)$. De aquí deducimos que lo importante de esta noción no son los valores exactos que puedan tomar $\delta$ y $\varepsilon$ sino la existencia de esas cantidades positivas. La idea que subyace a la noción de continuidad es que $f$ lleva el entorno de un punto $x$ ($B(x, \delta)$) en un entorno de $f(x)$ ($B(f(x), \varepsilon)$), es decir, lleva puntos cercanos de $x$ en puntos cercanos de su imagen $f(x)$. Esta abstracción nos va a permitir dar una definición de continuidad en espacios topológicos sin necesidad de utilizar distancias.

\begin{definition}Una aplicación $f \colon (X, \tau_X) \longrightarrow (Y, \tau_Y)$ entre dos espacios topológicos se dice \textbf{continua} respecto de $\tau_X$ y $\tau_Y$, ó \textbf{$(\tau_X, \tau_Y)$-continua, en un punto $x_0 \in X$} si para todo abierto $V$ en $Y$ tal que $f(x_0) \in V$ existe un abierto $U$ en $X$ tal que $x_0 \in U$ y $f(U) \subset V$. Diremos que $f$ es \textbf{$(\tau_X, \tau_Y)$-continua} si lo es en todo punto de $X$.
\end{definition}
\begin{center}
\begin{tikzpicture}
\draw (0,0) rectangle (3,4) node[above] at(1,4.1) {$X$};
\draw (1.6,2.2) ellipse (1.2 and 1.4);
\fill (2,2.2) circle (1pt) node [below] {$x_0$};
\draw (1.6,2.2) ellipse (0.7 and 0.9) node[above] at(1,2.7) {$U$};
\draw (10,0) rectangle (13,4) node[above] at(11,4.1) {$Y$};
\draw (11.5,2.2) ellipse (1.3 and 1.5) node[above] at(10.7,3.3) {$V$};
\draw (11.4,2.0) ellipse (0.7 and 0.9) node[above] at(11.8,2.75) {$f(U)$};
\fill (11.5,2.2) circle (1pt) node [below] {$f(x_0)$};
\draw [dashed, ->] (3,2.2)--(10,2.2) node [pos=0.5,above] {$f$};
\end{tikzpicture}
\end{center}

En el par $(X, \tau)$ es más relevante la topología $\tau$ que el conjunto $X$, también las topologías son esenciales en el concepto de continuidad, sin embargo, de ahora en adelante denotaremos $X$ por $(X, \tau)$, $Y$ por $(Y, \tau_Y)$ y $f \colon X \longrightarrow Y$ por $f \colon (X, \tau_X) \longrightarrow (Y, \tau_Y)$. Análogamente, diremos que $f$ es continua en lugar de $(\tau_X, \tau_Y)$-continua cuando las topologías $\tau_X, \tau_Y$ estén prefijadas o se sobreentiendan y no haya lugar a la confusión.

Por ejemplo, toda aplicación constante, independientemente de las topologías prefijadas en los espacios, es continua: en efecto, sea $f \colon X \longrightarrow Y$ definida por $f(x) = y_0$ para todo $x \in X$, dado un entorno $V$ de $y_0$ tomamos $U = X$, el cual es abierto para toda la topología sobre $X$, y es claro que $f(X) = \lbrace y_0 \rbrace \subset V$.

La condición de continuidad es suficiente definirla para abiertos básicos: si $\mathfrak{B}_Y$ es una base de $\tau_Y$ y $V \in \tau_Y$, entonces $V = \cup V_i$, con $V_i \in \mathfrak{B}_Y$ y si para cada $V_i$ existe un $U_i \in \tau_X$ tal que $f(U_i) \subset V_i$, es claro que $U = \cup U_i$ es un abierto en $X$ tal que $f(U) \subset V$, ya que $f(\cup U_i) = \cup f(U_i)$.

\begin{proposition}Dada $f \colon X \longrightarrow Y$, son equivalentes:
\begin{enumerate}
\item $f$ es continua.
\item $f^{-1}(U) \in \tau_X$ para todo $U \in \tau_Y$.
\item $f^{-1}(F)$ cerrado en $X$ para todo $F$ cerrado en $Y$.
\item $f(\overline{A}) \subset \overline{f(A)}$ para todo $A \subset X$.
\end{enumerate}
\end{proposition}
\begin{proof}
Sea $V \in \tau_Y$, dado $x \in f^{-1}(V)$ es claro que $f(x) \in V$, entonces si $f$ es continua existirá $U \in \tau_X$ tal que $x \in U$ y $f(U) \subset V$, luego $x \in U \subset f^{-1}(V)$ y así $f^{-1}(V) \in \tau_X$.

Sea $F$ cerrado en $Y$, luego $Y\setminus F \in \tau_Y$ y por tanto $X \setminus f^{-1}(F) = f^{-1}(Y \setminus F) \in \tau_X$, por lo que $f^{-1}(F)$ es cerrado en $X$.

Dado $A \subset X$, notar que $\overline{f(A)}$ es la intersección de todos los cerrados que contienen a $f(A)$, si $F$ es un cerrado en $Y$ tal que $f(A) \subset F$, entonces $A \subset f^{-1}(F)$ y $f^{-1}(F)$ cerrado implica que $\overline{A} \subset f^{-1}(F)$ ó bien $f(\overline{A}) \subset F$, por lo que $f(\overline{A}) \subset \overline{f(A)}$. 

Dado $x \in X$ y un entorno $V$ de $f(x)$, tomamos $U = X\setminus \overline{X\setminus f^{-1}(V)}$, y así $U$ es abierto y $x \in U$: en efecto, si $x \in \overline{X \setminus f^{-1}(V)}$ se sigue de $4$ que $f(x) \in f(\overline{X\setminus f^{-1}(V)} ) \subset f(\overline{X\setminus f^{-1}(V)})$, en particular $V \cap f(X \setminus f^{-1}(V)) \neq \emptyset$, sea $y \in V \cap f(X \setminus f^{-1}(V))$, entonces existe $z \in X \setminus f^{-1}(V)$ tal que $f(z) = y \in V$ y por tanto $z \in f^{-1}(V)$, lo cual es absurdo, luego $x \in X \setminus \overline{X\setminus f^{-1}(V)} = U$, además es claro que $U = X\setminus \overline{X\setminus f^{-1}(V)} \subset X \setminus (X\setminus f^{-1}(V)) = f^{-1}(V)$, por lo que $f(U) \subset V$ y concluimos que f es continua. 

\end{proof}

Notar que, como los abiertos y los cerrados se determinan mutuamente, el punto $3$ es equivalente a dar una condición suficiente para decidir la continuidad: \textbf{\textit{si $V$ es un abierto en $Y$ entonces $f^{-1}(V)$ es abierto en $X$.}}

\begin{example}Cualquier aplicación entre un espacio topológico discreto $(X, \tau_D)$ y un espacio topológico arbitrario $(Y, \tau)$ $$
\begin{array}{rccl}
f\colon &(X, \tau_D)& \longrightarrow & (Y, \tau)
\end{array}
$$ es continua, ya que si $V$ es abierto en $Y$, $f^{-1}(V) \subset X$, por lo que es abierto en $X$.

Por otro lado, cualquier aplicación de un espacio topológico arbitrario $(X, \tau)$ a un espacio topológico trivial con la topología indiscreta $(Y, \tau_I)$ $$
\begin{array}{rccl}
f\colon &(X, \tau)& \longrightarrow & (Y, \tau_I)
\end{array}
$$ es continua. Esto es así ya que los únicos abiertos de $Y$ son el propio $Y$ y el vacío, cuyas imágenes inversas son respectivamente el conjunto $X$ y el vacío, que son abiertos en $X$.
\end{example}

$\hfill \blacksquare$

\begin{proposition}Sea $X$ un conjunto y $\tau$, $\tau'$ dos topologías en $X$. Entonces, la aplicación identidad $$
\begin{array}{rccl}
id\colon &(X, \tau)& \longrightarrow & (X, \tau')\\
&x& \longmapsto &x
\end{array}
$$ es continua si y sólo si $\tau$ es más fina que $\tau'$.
\end{proposition}
\begin{proof}
La aplicación $id$ es continua si y sólo si dado $V \in \tau'$, $id^{-1}(V) = V \in \tau$, ó equivalentemente si $\tau' \subset \tau$.

\end{proof}

\begin{definition}Sea $X$ un conjunto, dada una aplicación $f \colon (X, \tau_X) \longrightarrow (Y, \tau_Y)$, es claro que $\tau_f = \lbrace f^{-1}(V): V \in \tau_Y\rbrace$ es una topología sobre $X$ y es la menor de todas las que hacen continua a $f$. La denominaremos \textbf{topología débil} inducida por $f$.
\end{definition}

\begin{definition}Dado un espacio topológico $(X, \tau)$ y $A \subset X$ definimos la \textbf{topología relativa} $\tau_A$ sobre $A$ como la topología débil inducida por la inclusión $i\colon A \longrightarrow X$, es decir, $\tau_A = \lbrace V \cap A :V \in \tau \rbrace$, ya que $i ^{-1}(V) = V \cap A$. Así, es claro que $U \subset A$ es abierto en $A$ si y sólo si $U = V \cap A$ para algún $V$ abierto en $X$. El par $(A, \tau_A)$ se dirá \textbf{subespacio} de $(X, \tau)$.
\end{definition}

Notar que $\tau_A$ es la menor topología sobre $A$ que hace continua la inclusión. En concreto, si $A$ es abierto en $X$ y $B \subset A$ también es abierto en $A$ entonces $B$ es abierto en $X$.

\begin{proposition}Sea $(X, \tau)$ un espacio topológico y $(A, \tau_A)$ un subespacio, entonces $B \subset A$ es cerrado en $A$ si y sólo si $B = F \cap A$, con $F$ cerrado en $X$.
\end{proposition}
\begin{proof}
Si $B$ es cerrado en $A$ entonces $A \setminus B \in \tau_A$, es decir, $A \setminus B = U \cap A$, con $U \in \tau$, pero $B = A \setminus (U \cap A) = (X \setminus U ) \cap A$ y tomamos $F = X \setminus U.$

Recíprocamente, si $B = F \cap A$, con $F$ cerrado en $X$, entonces $U = X \setminus F$ es abierto en $X$ y $A \setminus B = A \setminus (F \cap A) = A \setminus [(X \setminus U) \cap A] = U \cap A$ abierto en $A$, por lo que $B$ es cerrado en $A$.

\end{proof}

Como consecuencia inmediata se sigue que si $B \subset A$ es cerrado en $A$ y $A$ es cerrado en $X$ entonces también $B$ es cerrado en $X$.

\begin{example}
Sea $A = [0,1) \cup \lbrace 2 \rbrace \subset \mathbb{R}$ con la topología relativa inducida por la usual de $\mathbb{R}$, entonces $[0,1) = (-1, 1) \cap A = [0,1] \cap A$ y $\lbrace 2 \rbrace = (1,3) \cap A = [2,3] \cap A$ son abiertos y cerrados en $A$ pero $A$ no es abierto ni cerrado en $\mathbb{R}$.
\end{example}
$\hfill \blacksquare$

\begin{proposition}Sea $B \subset A \subset X$, si $Cl_A B$ denota la clausura de $B$ en $A$, entonces $Cl_A B = A \cap \overline{B}$.
\end{proposition}
\begin{proof}
Ya que $A \cap \overline{B}$ es el menor cerrado en $A$ que contiene a $B$. 

\end{proof}

\begin{proposition}La restricción en el dominio o en la imagen de aplicaciones continuas también son continuas.
\end{proposition}
\begin{proof}
Sea $f \colon X \longrightarrow Y$ continua, $A \subset X$ y $i \colon A \longrightarrow X$ la inclusión, entonces $\left.f \right|_A \colon A \longrightarrow Y$ es continua ya que $\left.f \right|_A = fi$ y la composición de aplicaciones continuas es continua. 

Sea $g \colon X \longrightarrow f(X)$ la restricción de $f$ en la imagen, entonces $f = ig$, donde $i$ es la inclusión de $f(X)$ en $Y$, si $U$ es abierto en $f(X)$ notar que $U = V \cap f(X) = i^{-1}(V)$, con $V$ abierto en $Y$, entonces $g^{-1}(U) = g^{-1}i^{-1}(V) = (ig)^{-1}(V) = f^{-1}(V)$, abierto por ser $f$ continua, y se sigue por tanto que $g$ es continua.

\end{proof}

\begin{definition}Un \textbf{recubrimiento abierto} de un espacio topológico $X$ es una colección de abiertos $\mathcal{U} = \lbrace U_i \rbrace_{i \in J}$ tales que $X = \cup_{i \in J}U_i$.  
\end{definition}

Sea $i_k \colon U_k \longrightarrow X$. Denotaremos por $f_k = fi_k$ a la restricción de $f$ a $U_k$.

\begin{proposition}Sea $\mathcal{U} = \lbrace U_i \rbrace_{i\in J}$ un recubrimiento abierto de $X$, entonces $f\colon X \longrightarrow Y$ es continua si y sólo si lo es $f_k \colon U_k \longrightarrow Y$ para todo $k \in J$.
\end{proposition}
\begin{proof}
Si $f$ es continua también lo son las restricciones $f_k = fi_k$ para todo $k \in J$.

Recíprocamente, sea $f_k$ continua para todo $k \in J$ y $V$ un abierto en $Y$, entonces $f^{-1}(V) = f^{-1}(V) \cap (\cup U_k) = \cup [f^{-1}(V) \cap U_k] = \cup i_k^{-1}[f^{-1}(V)] = \cup (fi_k)^{-1}(V) = \cup f_k^{-1}(V)$, pero $f_k^{-1}(V)$ es abierto para todo $k \in J$.
\end{proof}

\begin{proposition}Sean $A$ y $B$ dos cerrados en $X$ tales que $X = A \cup B$, dadas dos aplicaciones continuas $f_A \colon A \longrightarrow Y$ y $f_B \colon B \longrightarrow Y$ tales que $f_A(x) = f_B(x)$ para todo $x \in A \cap B$, entonces la aplicación $f \colon X \longrightarrow Y$ dada por $f(x) = f_A(x)$ si $x \in A$ y $f(x) = f_B(x)$ si $x \in B$ también es continua.
\end{proposition}
\begin{proof}
Sea $F$ cerrado en $Y$, entonces $f_A^{-1}(F) = f^{-1}(F) \cap A$ cerrado en $A$ y, por ser $A$ cerrado, también cerrado en $X$. Análogamente $f_B ^{-1} (F) = f^{-1}(F) \cap B$ cerrado en $B$ y en $X$, pero $f^{-1}(F) = f^{-1}(F) \cap X = f^{-1}(F) \cap (A\cup B) = (f^{-1}(F) \cap A)\cup (f^{-1}(F) \cap B) = f_A^{-1}(F) \cup f^{-1}_B(F)$ cerrado y por tanto $f$ continua.

\end{proof}

\begin{definition}Una aplicación $f \colon X \longrightarrow Y$ se dice \textbf{abierta} si $f(A)$ es abierto en $Y$ para todo $A$ abierto en $X$. Análogamente para las \textbf{cerradas}.
\end{definition}

\begin{example}Veamos algunos ejemplos:
\begin{enumerate}
\item Una aplicación abierta o cerrada no tiene por qué ser continua: si $\tau_1 \subset \tau_2$, entonces $id_X \colon (X, \tau_1) \longrightarrow (X, \tau_2)$ es abierta y cerrada pero no es continua si el contenido es estricto.
\item Sea $i \colon A \longrightarrow X$ la inclusión, entonces $i$ es continua, pero es abierta (o cerrada) si y sólo si $A$ es abierto (cerrado) en $X$.
\item Si $f \colon \mathbb{R} \longrightarrow \mathbb{R}$ es una aplicación constante, entonces es continua y cerrada pero no abierta.
\item La aplicación $f\colon \mathbb{R} \longrightarrow \mathbb{R}$ dada por $f(x) = x^2$ no es abierta: en efecto, $f((-1,1)) = [0,1)$.
\item Si $f$ es una biyección entonces $f(X \setminus A) = Y \setminus f(A)$ para todo $A \subset X$, luego $f$ es abierta si y sólo si es cerrada.
\item La aplicación $f \colon \mathbb{R} \longrightarrow \mathbb{R}$ dada por $f(x) = \dfrac{1}{1+x^2}$ es continua pero no es abierta ni cerrada. Esto es así ya que $f((-1,1))=(1/2,1]$ no es abierto y $f(\mathbb{R}) = (0,1]$ no es cerrado.
\end{enumerate}
\end{example}

$\hfill \blacksquare$

\begin{proposition}Una aplicación $f \colon X \longrightarrow Y$, no necesariamente continua, es cerrada si y sólo si $\overline{f(A)} \subset f(\overline{A})$, para todo $A \subset X$.
\end{proposition}
\begin{proof}
Sea $f$ cerrada y $A \subset \overline{A}$, entonces $f(A) \subset f(\overline{A})$, el cual es cerrado, luego $\overline{f(A)} \subset f(\overline{A})$ ya que $\overline{f(A)}$ es el menor cerrado que contiene a $f(A)$.

Recíprocamente, si se verifica la relación de contenido y $F$ es cerrado en $X$, entonces $\overline{f(F)} \subset f(\overline{F}) = f(F)$ ya que $\overline{F} = F$, por lo que $\overline{f(F)} = f(F)$, es decir, $f(F)$ cerrado y así $f$ es cerrada.

\end{proof}

Una útil caracterización de las aplicaciones cerradas es la siguiente: 

\begin{theorem}
Sea $f \colon X \longrightarrow Y$ una aplicación no necesariamente continua, entonces $f$ es cerrada si y sólo si para todo $B \subset Y$ y todo $U \in \tau_X$ tal que $f^{-1}(B) \subset U$ existe $V \in \tau_Y$ tal que $B \subset V$ y $f^{-1}(V) \subset U$.
\end{theorem}
\begin{proof}
Sea $f$ cerrada, $B \subset Y$ y $U$ un abierto en $X$ tal que $f^{-1}(B) \subset U$, entonces es claro que $V = Y \setminus f(X\setminus U)$ es abierto, veamos que es el abierto buscado: $f^{-1}(V) = f^{-1}(Y \setminus f(X\setminus U)) = X \setminus f^{-1}f(X\setminus U) \subset X \setminus (X \setminus U) = U$; por otra parte sea $y \in B$, si $f^{-1}(y) \neq \emptyset$ es claro que $y \in V$, suponemos entonces que $f(y) \neq \emptyset$, como $f^{-1}(y) \subset f^{-1}(B) \subset U$ se sigue que $f^{-1}(y) \cap (X \setminus U) = \emptyset$ luego $y= ff^{-1}(y) \in Y \setminus f(X \setminus U) = V$, en todo caso puesto que $y \in V$.

Recíprocamente, sea $F$ cerrado en $X$ y $B = Y \setminus f(F)$, entonces $f^{-1}(B) = f^{-1}[Y \setminus f(F)] = X \setminus f^{-1}f(F) \subset X \setminus F$ abierto, luego si tomamos $U = X \setminus F$, aplicando la hipótesis existirá $V$ abierto en $Y$ tal que $Y \setminus f(F) \subset V$ y $f^{-1}(V) \subset X \setminus F$, notar que $y \in V$ implica que $f^{-1}(y) \subset f^{-1}(V) \subset X \setminus F$, o bien $f^{-1}(y)\cap F = \emptyset$, luego $y \notin f(F)$, es decir,  $y \in Y \setminus f(F)$, y así $V \subset Y \setminus f(F)$ y como teníamos el otro contenido se sigue la igualdad $V = Y \setminus f(F)$, pero $V$ abierto implica que $f(F)$ es cerrado y así $f$ es cerrada.

\end{proof}

Ya hemos hablado acerca de las aplicaciones continuas, que preservan la estructura de espacio topológico. Ahora daremos una definición de lo que se considera como \textit{equivalencia topológica}, que formaliza la idea intuitiva de deformación de nuestro espacio, es decir, las alteraciones que no cortan ni pegan nuestro espacio preservan la topología.

\begin{definition}Una biyección continua $f \colon X \longrightarrow Y$ entre dos espacios topológicos se dirá \textbf{homeomorfismo} o \textbf{equivalencia topológica} si también su inversa $f^{-1}$ es continua. Si existe tal homeomorfismo diremos que $X$ e $Y$ son homeomorfos, y lo denotaremos $X \cong Y$.
\end{definition}

Una aplicación continua, como ya sabemos, lleva puntos cercanos en puntos cercanos, pero un homeomorfismo es una aplicación que también lleva los puntos lejanos a puntos lejanos.

Veamos una caracterización de los homeomorfismos:

\begin{theorem}
Sea $f \colon X \longrightarrow Y$ biyectiva, son equivalentes: 
\begin{enumerate}
\item $f^{-1}$ es continua.
\item $f$ es abierta.
\item $f$ es cerrada.
\end{enumerate}

Es decir, diremos que $f$ es un homeomorfismo si y sólo si se cumple cualquiera de las condiciones equivalentes anteriores.
\end{theorem}
\begin{proof}
Veamos que $1.$ implica $2.$. Supongamos que $f^{-1}$ es continua y sea $U \subset X$ un abierto. Entonces $(f^{-1})^{-1}(U)$ es abierto y, como $(f^{-1})^{-1}(U) = f(U)$ se sigue que $f(U)$ es abierto en $Y$.

Veamos ahora que $2.$ implica $3.$. Supongamos que $f$ es abierta y sea $C \subset X$ un cerrado. Entonces $X \setminus C$ es abierto, y como $f$ es abierta entonces $f(X \setminus C)$ es abierto en $Y$. Como $f$ es biyectiva tenemos que $f(X \setminus C) = Y \setminus f(C)$, así que $f(C)$ es cerrado y $f$ es cerrada.

Finalmente, veamos que $3.$ implica $1.$. Supongamos que $f$ es cerrada. Para ver que $f^{-1}$ es continua nos bastará comprobar que, dado un cerrado $C$ en $X$, $(f^{-1})^{-1}(C)$ es cerrado en $Y$. Como $f$ es cerrada y $f=(f^{-1})^{-1}(C)$ se tiene que $(f^{-1})^{-1}(C) = f(C)$, que es cerrado en $Y$.

\end{proof}

\begin{example}Un ejemplo clásico es el de los números reales $\mathbb{R}$. Y es que el conjunto de los números reales $\mathbb{R}$ con la topología usual $\tau_u$, es decir, la recta real, es homeomorfo a cualquier intervalo abierto con la topología relativa sobre ese intervalo. 

Veamos tres tipos de homeomorfismos que al componerlos dan el resultado: 
\begin{enumerate}
\item Sea $(a,b)$ con $a,b \in \mathbb{R}$, con $a <b$, un intervalo de $\mathbb{R}$ dotado de la topología relativa. La aplicación  $$
\begin{array}{rccl}
h_1\colon &(0, 1)& \longrightarrow & (a, b)\\
&x& \longmapsto &(b-a)x+a
\end{array}
$$ es continua y su inversa $$h_1^{-1}(y) = \dfrac{y-a}{b-a}$$ es también continua. Luego, $h_1$ es un homeomorfismo entre $(0,1)$ y $(a,b)$.
\item La aplicación $$
\begin{array}{rccl}
h_2\colon &\mathbb{R}& \longrightarrow & (0, + \infty)\\
&x& \longmapsto &e^x
\end{array}
$$ es continua y su inversa $$h_2^{-1}(y) = ln y$$ es también continua. Así, $h_2$ es un homeomorfismo entre $\mathbb{R}$ y el intervalo $(0, + \infty)$.
\item La aplicación $$
\begin{array}{rccl}
h_3\colon &\mathbb{R}& \longrightarrow & (-\pi/2, \pi/2)\\
&x& \longmapsto &\arctan x
\end{array}
$$ y su inversa $$h_3^{-1}(y) = \tan y$$ son continuas, por lo que $h_3$ establece un homeomorfismo entre $\mathbb{R}$ y el intervalo $(-\pi/2,\pi/2)$.
\end{enumerate}
Y ahora, componiendo estos homeomorfismos se sigue que cualquier intervalo $(a,b) \subset \mathbb{R}$ es homeomorfo a $\mathbb{R}$.
\end{example}

$\hfill \blacksquare$

\begin{example}$\mathbb{R}^n$ con la topología usual es homeomorfo a la bola unidad  centrada en el origen $B(0,1)$. La aplicación  $$
\begin{array}{rccl}
h\colon &\mathbb{R}^n& \longrightarrow & B(0, 1)\\
&x& \longmapsto &\dfrac{x}{1+\norm{x}}
\end{array}
$$ es una aplicación continua cuya inversa  $$
\begin{array}{rccl}
h{-1}\colon &B(0, 1)& \longrightarrow & \mathbb{R}^n\\
&y& \longmapsto &\dfrac{y}{1-\norm{y}}
\end{array}
$$ es también continua.
\end{example}

$\hfill \blacksquare$

\begin{example}En $\mathbb{R}^{n+1}$ consideramos el subespacio $$S^n = \lbrace (x_1, \ldots, x_{n+1} \in \mathbb{R}^{n+1}:x_1^2+\ldots +x_{n+1}=1\rbrace.$$ Este subespacio $S^n$ se denonima \textbf{esfera unidad $n$-dimensional}. Definimos el \textbf{polo norte} de la esfera como el punto $P_N = (0, \ldots, 0, 1)$. Entonces la aplicación $$
\begin{array}{rccl}
f\colon &S^n\setminus \lbrace P_N \rbrace& \longrightarrow & \mathbb{R}^n\\
&(x_1, \ldots, x_{n+1})& \longmapsto &\dfrac{1}{1-x_{n+1}}(x_1,  \ldots, x_n), 
\end{array}
$$donde $h(x)$, $x \in \mathbb{R}^n$, es el punto de intersección de la recta que une el polo norte con $x$ y el plano $x_{n+1}=0$, es un homeomorfismo que denominaremos \textbf{proyección estereográfica}.
\end{example}

$\hfill \blacksquare$

\subsection{Topología producto y topología cociente}
Recordemos que, dados dos conjuntos $A,B$, podemos definir su producto cartesiano como el conjunto de los pares de elementos, donde el primero pertenece a $A$ y el segundo a $B$. Es decir, $A \times B = \lbrace (a,b): a \in A, b \in B \rbrace$. También recordemos que podemos definir las proyecciones $p_A$ y $p_B$ como las aplicaciones $$
\begin{array}{rccl}
p_A\colon &A\times B& \longrightarrow & A\\
&(a,b)& \longmapsto &a
\end{array}
$$ $$
\begin{array}{rccl}
p_B\colon &A\times B& \longrightarrow & B\\
&(a,b)& \longmapsto &b
\end{array}
$$ 
Dados dos espacios topológicos $(X, \tau_X), (Y, \tau_Y)$ queremos definir sobre el produco cartesiano $X \times Y$ una topología tal que las proyecciones canónicas $p_X, p_Y$ sean continuas, es decir, $p_X^{-1}(U) = U \times Y$ y $p_Y^{-1}(V) = X \times V$ deben ser abiertos en $X \times Y$ para todo $U \in \tau_X$, $V \in \tau_Y$.

\begin{definition}Dados dos espacios topológicos $(X, \tau_X), (Y, \tau_Y)$, su producto cartesiano $X \times Y$ y las proyecciones $p_X \colon X \times Y \longrightarrow X$, $p_Y \colon X\times Y \longrightarrow Y$, entonces la menor topología sobre $X \times Y$ que hace continuas las proyecciones es la que tiene como base a $\mathfrak{B}_p = \lbrace U \times V: U \in \tau_X, V \in \tau_Y \rbrace$. Esta topología, $\tau_p$, la denominaremos \textbf{topología producto}. 
\end{definition}

Notar que una subbase para $\tau_p$ es $\mathcal{S}_p = \lbrace U \times Y: U \in \tau_X \rbrace \cup \lbrace X \times V: V \in \tau_Y \rbrace$.

Dadas $f \colon Z \longrightarrow X$ y $g \colon Z \longrightarrow Y$, existe una única aplicación $h \colon Z \longrightarrow X \times Y$ satisfaciendo $p_Xh = f$ y $p_Yh=g$, es lo que se conoce como propiedad universal del producto directo. Notar que $h(z) = (f(z),g(z))$ y es usual denotar $h=(f,g)$.

\begin{proposition}La aplicación $h = (f,g)$ es continua si y sólo si $f$ y $g$ lo son. En particular, la diagonal $\Delta = (id_X, id_X) \colon X  \longrightarrow X \times X$ es continua.
\end{proposition}
\begin{proof}
Si $h$ es continua, $f = p_Xh$ y $g = p_Yh$ también lo son, por serlo $p_X$ y $p_Y$. 

Recíprocamente, si $f$ y $g$ son continuas y $U \times V$ es un abierto básico, $h^{-1}(U \times V) = h^{-1}(U \times Y \cap X \times V) = h^{-1}(p_X^{-1}(U) \cap p_Y^{-1}(V)) = h^{-1}p_X^{-1}(U) \cap h^{-1}p_Y^{-1}(V) = (p_Xh)^{-1}(U) \cap (p_Yh)^{-1}(V) = f^{-1}(U) \cap g^{-1}(V)$ abierto en $Z$, luego $h$ es continua.

\end{proof}

\begin{definition}Sea $X$ un conjunto, dada una familia de espacios topológicos $\lbrace (X_i, \tau_i) \rbrace_{i\in J}$ y una familia de aplicaciones $\lbrace f_i \colon X \longrightarrow X_i \rbrace_{i\in J}$ definimos la \textbf{topología débil} sobre $X$ inducida por la familia $\lbrace f_i \rbrace_{i \in J}$ como la menor topología que hace continuas las $f_i$, es decir, $\mathcal{S} = \cup_{i \in J} S_i$, con $S_i = \lbrace f_i^{-1}(U_{ij}) : U_{ij} \in \tau_i \rbrace$ es una subbase para dicha topología y una base para la misma estará formada por las intersecciones finitas de elementos de $\mathcal{S}$.
\end{definition}

\begin{theorem}
Sea $X$ con la topología débil inducida por la familia $\lbrace f_i \rbrace_{i \in J}$, entonces $h \colon Y \longrightarrow X$ es continua si y sólo si $f_ih$ es continua para todo $i \in J$.
\end{theorem}
\begin{proof}
Si $h$ es continua, entonces $f_ih$ es continua para todo $i \in J$, ya que las $f_i$ son continuas. 

Recíprocamente, sea $U$ abierto en $X$, entonces $U$ es unión arbitraria de intersecciones finitas de elementos de la forma $f^{-1}_i(U_{ij})$, por tanto $h^{-1}(U)$ será unión arbitraria de intersecciones finitas de elementos de la forma $h^{-1}f_i^{-1}(U_{ij}) = (f_ih)^{-1}(U_{ij})$, abiertos en $Y$ si las $f_ih$ son continuas para todo $i \in J$, luego $h^{-1}(U)$ abierto y por tanto $h$ continua.

\end{proof}

Una forma alternativa de ver la topología producto es la que sigue. Sea $X = \prod_{i\in J} X_i$, un punto del producto será una $|J|$-tupla, $(x_i)$. Además, denotamos por $p_k \colon \prod X_i \longrightarrow X_k$ dada por $p_k ((x_i)) = x_k$ la proyección canónica sobre $X_k$, entonces definimos la \textbf{topología producto}, $\tau_p$, sbre $\prod X_i$ como la topología débil inducida por las proyecciones $\lbrace p_i \rbrace_{i \in J}$.

Si $U \in \tau_k$ notar que $p_k^{-1}(U) = \prod U_i$, donde $U_k = U$ y $U_i = X_i$ para todo $i \neq k$. Una subbase de $\tau_p$ viene dada por $\mathcal{S}_p = \lbrace p_i^{-1}(U) : U \in \tau_i, i \in J \rbrace$ y notar que si $\mathfrak{B}_p$ es la base de $\tau_p$ definida por $\mathcal{S}_p$, entonces $\prod U_j \in \mathfrak{B}_p$ si y sólo si $U_j = X_j$ para todo $j \in J \setminus F$, donde $F \subset J$ es un subconjunto finito de índices.

Dada una familia $\lbrace f_i \colon X \longrightarrow X_i \rbrace_{i \in J}$, la aplicación $f \colon X \longrightarrow \prod X_i$ dada por $f(x) = (f_i(x))$ es única satisfaciendo $p_if = f_i$ para todo $i \in J$.

\begin{corolario}$f$ es continua si y sólo si lo es $f_i = p_if$ para todo $i \in J$.
\end{corolario}

\begin{proposition}Sea $\lbrace f_i \colon X_i \longrightarrow Y_i \rbrace _{i\in J}$ una familia de aplicaciones y definimos $f \colon \prod X_i \longrightarrow \prod Y_i$ dado por $f((x_i)) = (f_i(x_i))$, es decir, $f = \prod f_i$, entonces $f$ es continua si y sólo si $f_i$ es continua para todo $i \in J$.
\end{proposition}
\begin{proof}
Es claro que $p_kf = f_kp_k$ para todo $k \in J$, entonces si las $f_k$ son continuas también lo es $f$. 

Recíprocamente, si $f$ es continua y $k \in J$ elegimos un punto $x_i^0 \in X_i$ para todo $i \neq k$ y podemos mirar a $X_k$ como subespacio del producto $\prod X_i$, es decir, $X_k \cong \prod Y_i$, con $Y_k = X_k$ y $Y_i = \lbrace x_i^0 \rbrace$ para todo $i \neq k$, entonces las inclusiones $i_k \colon X_k \longrightarrow \prod X_i$ dadas por $i_k(x) = (x_i)$ tales que $x_k = x$ y $x_i = x_i^0$ para todo $i \neq k$ son continuas, y es claro que $f_k = p_kfi_k$ para todo $k \in J$, por lo que $f_k$ es continua si lo es $f$.

\end{proof}
\section{Separación y numerabilidad}
\section{Espacios métricos}
\section{Compacidad}
\section{Conexión}

\end{document}